\documentclass[12pt, a4paper,bibliography=totoc, twoside,BCOR=12.5mm,abstracton]{scrartcl}

\usepackage[utf8]{inputenc}
\usepackage{ngerman}
\usepackage{graphicx}
\usepackage{float}
\usepackage{natbib} % Zitate
\bibliographystyle{apalike}
%\usepackage{hyperref} % eventuell am Schluss aktivieren
\usepackage{url}
\usepackage{array}
\usepackage{color}
\usepackage{listings}
\usepackage{caption}
\usepackage{amsmath} %fancy Mathematik umgebung
\usepackage[headsepline, plainheadsepline]{scrpage2}
\usepackage{todonotes}

% optionen fuer todonotes
% \todo
% \listoftodos
% \missingfigure




%Optionen für das Paket listings
\definecolor{mygreen}{rgb}{0,0.6,0}
\definecolor{mygray}{rgb}{0.5,0.5,0.5}
\definecolor{mymauve}{rgb}{0.58,0,0.82}
\lstset{ 
  backgroundcolor=\color{white},   % choose the background color; you must add \usepackage{color} or \usepackage{xcolor}
  basicstyle=\footnotesize,        % the size of the fonts that are used for the code
  breaklines=true,                 % sets automatic line breaking
  captionpos=b,                    % sets the caption-position to bottom
  commentstyle=\color{mygreen},    % comment style
  extendedchars=true,              % lets you use non-ASCII characters; for 8-bits encodings only, does not work with UTF-8
  keepspaces=true,                 % keeps spaces in text, useful for keeping indentation of code (possibly needs columns=flexible)
  keywordstyle=\color{blue},       % keyword style
  language=C++,               	   % the language of the code
  numbers=left,                    % where to put the line-numbers; possible values are (none, left, right)
  numbersep=5pt,                   % how far the line-numbers are from the code
  numberstyle=\tiny\color{mygray}, % the style that is used for the line-numbers)
  stepnumber=1,                    % the step between two line-numbers. If it's 1, each line will be numbered
  stringstyle=\color{mymauve},     % string literal style
  tabsize=4,                       % sets default tabsize to 2 spaces
}


%Optionen für das Paket cations
\captionsetup{	margin	=10pt,
		font	=footnotesize,
		labelfont=bf,
		labelsep=colon,
		format	=plain
		}


%Kopfzeilen formatierung
\automark[section]{subsection}
\clearscrheadings
\ihead{}
\ohead{\rightmark}
\ofoot[scrplain-außen]{\thepage}

%Kopfzeile mit Kapitelangabe
\pagestyle{scrheadings} 

%Gleichungsnummerierung folgt Kapiteln
\numberwithin{equation}{section}
 
%keine Einrückung bei neuen Absätzen
\parindent 0pt

%Titelseite
%\title{Bildaufnahmen mit einer SO\textsubscript{2}-Kamera \\
%\small{Bachelor Arbeit}}
%\author{Morten Harms \\ Universität Hamburg \\ Institut für Geophysik}
%\date{\today}

\begin{document}
\title{Lösung Übung 6 Potentialtheorie}
\author{Henrik Grob \& Morten Harms}
\date{\today}
\maketitle

\section{Einleitung}
In dieser Übung wird das erstellen eines Programms zur Berechnung der Schwerebeschleunigung eines beliebig geformten Störkörpers beschrieben. Aus Grundlage dient ein Programm, welches Donald Plouff 1976 in einem Paper \citep{plouff:1976} beschrieben hat. Das Programm wurde in der Programmiersprache Fortran geschrieben. Zum erstellen der Grafiken wurde das Programm Gnuplot verwendet. Die Qualität der Ergebnisse wird am Ende dieser Übung diskutiert

\section{Funktionsweise}
Das Programm ist in der Lage, für ein beliebig geformten Störkörper, die Schwerebeschleunigung an jedem Punkt im Raum zu berechnen. Voraussetzung ist, dass der Störkörper durch \textit{k} \textit{n}-Eckige Polygone approximiert wird. In dem Paper \citep{talwani:1960}, auf welches sich Plouff bezieht wird die Schwerebeschleunigung eines Körpers bestimmt, indem der Körper durch Laminare angenähert wird. Plouff  nutzt eine analytische Lösung für das Integral nach der Tiefe um aus den Laminaren Polygone zu erzeugen, dies sich nach der Tiefe nicht veränderen. Es ist ihm damit möglich eine exakte Lösung für die Schwerebeschleunigung eines Polygons zu berechnen. Durch das geschickte Stapeln von Polygonen ist er in der Lage beliebig geformte Störkörper zu berechnen. Das Verfahren von Plouff hat den Vorteil, dass kein Fehler über die Integration entsteht. Der Fehler stammt nur aus der Approximation des Störkörpers durch \textit{n}-Eckige Polygone.

In der Subroutine \verb+calc_grav_polygon( ... )+ wird durch die Gleichung \ref{eq:plouff_3} der Anteil der Schwerebeschleunigung die jede Kante eines solchen Polygons zur Gesamtbeschleunigung beiträgt errechnet und aufsummiert.

In der Subroutine \verb+sum_grav ( ... )+ werden die Ergebnisse für alle Polygone aufsummiert. 

In der Subroutine \verb+profil_1d_x ( ... )+ Werden die Werte aus \verb+sum_grav ( ... )+ entlang der x-Achse ermittelt und rausgeschrieben.

Wie in der Aufgabenstellung gefordert wird das Programm getestet mit der Approximation einer Kugel. Die Koordinaten der Eckpunkte werden in der Subroutine \verb+create_sphere+ berechnet.
 
 \begin{equation}
 g = ...
 \label{eq:plouff_3}
 \end{equation}
 
\section{Ergebnisse}

\section{Qualität der Ergebnisse}





\bibliography{bibliography.bib}
 \end{document}